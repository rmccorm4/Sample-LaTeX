\documentclass[12pt]{amsart}

\usepackage{amsmath}
\usepackage{amssymb}
\usepackage{amsthm}
\usepackage{mathrsfs}
\usepackage{enumerate}

\newcommand{\N}{\mathbb{N}}
\newcommand{\C}{\mathbb{C}}
\newcommand{\R}{\mathbb{R}}
\newcommand{\Z}{\mathbb{Z}}
\newcommand{\Q}{\mathbb{Q}}

\setlength{\parindent}{0pt}
\setlength{\parskip}{1em}

\begin{document}

\title{5 10-24 McCormick Ryan}
\date{October 24, 2016}
\author{Ryan McCormick}

\maketitle

\theoremstyle{plain}
\newtheorem*{prop8.19}{Proposition 8.19}
\begin{prop8.19}
	\item[($\textbf{i}$)] $\forall m \in \R$, $-(-m)=m$.
	\item[($\textbf{ii}$)] $-0 = 0$.
\end{prop8.19}

\begin{proof}
	\item[($\textbf{i}$)] As shown in class, $-m=(-1)m$. \\
	Therefore $(-(-m)) = (-1)(-m)$. \\
	So we need to show that $(-1)(-m)=m$. \\
	Adding $-m$ to both sides, \\
	$(-1)(-m)+(-m)=m+(-m)$ \\
	For the right side, \\
	$m+(-m)=0$ (AXIOM 8.4) \\
	For the left side, \\
	$(-1)(-m)+(-m)=(-m)(-1 + 1)$ (AXIOM 8.1(iii)) \\
	$(-1 + 1) = (1 + -1) = 0$ (AXIOM 8.1(i) and Axiom 8.4) \\
	So now we have, \\
	$(-m)(-1 + 1) = (-m)(0) = 0$ (PROP 8.15) \\
	Which leaves us with both sides equaling 0, \\
	$0 = 0$, so both sides of the equation are equal to each other. $\checkmark$
	
	\item[($\textbf{ii}$)] As shown in class, $-m=(-1)m$. \\
	Therefore, $-0 = (-1)0$ and since $-1$ is the additive inverse of $1$, $-1 \in \R$ so $(-1)0 = 0$ by Prop 8.15, so $-0=0$.
\end{proof}

\newtheorem*{prop8.20}{Proposition 8.20}
\begin{prop8.20}
	Given $m,n \in \R$ there exists one and only one $x \in \R$ such that $m+x=n$.
\end{prop8.20}

\begin{proof}
	If $m=n$, then $x=0$ by Axiom 8.2, and $0$ is unique by Proposition 8.11. \\
	If $m \neq n$, then let there be two equations, $m+x_1 = n$ and $m+x_2 = n$. \\
	For the first equation, $x_1 = n + (-m)$ by adding $(-m)$ to both sides, and simplifying the equation using Axiom 8.1(i), Axiom 8.4, and Axiom 8.2. \\
	The same thing applies for the second equation, leaving us with $x_2 = n + (-m)$, therefore $x_1 = x_2$, so there only exists one $x \in \R$ that satisfies this equation for a given $m,n \in \R$.
\end{proof}

\newtheorem*{prop8.25}{Project 8.25}
\begin{prop8.25}
	Think about why division by 0 ought not to be defined. Come up with an argument that will convince a friend.
\end{prop8.25}

\begin{proof}
	Division is defined as $y/x$ or $y \cdot x^{-1}$.\\
	For a number divided by 0, $x=0$. \\
	However, in Axiom 8.5 which defines multiplicative inverses, $0$ is not defined to have a multiplicative inverse. Therefore it follows that division by 0 is not defined because you would be multiplying a number by the multiplicative inverse of 0, which is not defined.
\end{proof}

\newtheorem*{prop8.40}{Proposition 8.40}
\begin{prop8.40}
	\item[($\textbf{i}$)] $x \in R_{>0} \iff 1/x \in \R_{>0}$.
\end{prop8.40}

\begin{proof}
	($=>$)\\
	Since $x \in \R_{>0}, x\in \R$, so $x^{-1} \in \R$ by Axiom 8.5. \\
	Also, $x \cdot x^{-1} = 1$ by Axiom 8.5 and $1 \in \R_{>0}$ by Prop 8.28. \\
	Therefore since $x \in \R_{>0}, x^{-1} \in \R$, and $1 \in \R_{>0}$,then $x^{-1} \in \R_{>0}$ by Proposition 8.36. \\\\
	($<=$) \\
	$(1/x)^{-1} = x$ because $(1/x)\cdot(1/x)^{-1} = 1 = (1/x)\cdot x$. (Axiom 8.5)\\
	Since $1/x \in \R_{>0}, 1/x\in \R$, so $(1/x)^{-1} = x \in \R$ by Axiom 8.5. \\
	Also, $x^{-1}\cdot x = x \cdot x^{-1} = 1$ by Axiom 8.5 and $1 \in \R_{>0}$ by Prop 8.28. \\
	Therefore since $1/x \in \R_{>0}, (x) \in \R$, and $1 \in \R_{>0}$, then $x \in \R_{>0}$ by Proposition 8.36.
\end{proof}

\newtheorem*{prop8.41}{Proposition 8.41}
\begin{prop8.41}
	$x^2 < x^3 \iff x > 1$
\end{prop8.41}

\begin{proof}
	($=>$)\\
	$x^2<x^3$ \\
	$x \cdot x < x \cdot x \cdot x$ \\
	Multiplying by $x^{-1}$ on both sides, \\
	$x \cdot x \cdot x^{-1} < x \cdot x \cdot x \cdot x^{-1}$ \\
	$x \cdot 1 < x \cdot x \cdot 1$ (AXIOM 8.5)\\
	$x < x \cdot x$ (AXIOM 8.3) \\
	Multiplying by $x^{-1}$ on both sides again, \\
	$x \cdot x^{-1} < x \cdot x \cdot x^{-1}$ (AXIOM 8.3) \\
	$1 < x \cdot 1$ (AXIOM 8.5) \\
	$1 < x$ (AXIOM 8.3)\\
	$x > 1$ $\checkmark$ \\\\
	
	($<=$)\\
	$x>1 \implies 1<x$ \\
	Multiplying by $x^2$ on both sides, \\
	$1 \cdot x^2 < x \cdot x^2$ \\
	$x^2 < x^3$ (AXIOM 8.3)
\end{proof}

\newtheorem*{prop8.43}{Proposition 8.43}
\begin{prop8.43}
	Let $x,y \in \R$ such that $x < y$. There exists $z \in \R$ such that $x < z < y$.
\end{prop8.43}

\begin{proof}
	Assume for contradiction that $m = min(\R)$. Then $m-1 \in \R$, but $m-1 < m$ because $m-(m-1) = 1 \in \R_{>0}$. Therefore $\R$ has no minimum element. \\
	So for $x,y,z \in \R$, there exists a $z < y$ let's say $z=y-1$ for example since there is no minimum element. Similarly, there exists an $x < z$, let's say $x = z-1 = y-1-1$ for example for the same reason that there is always an real number smaller than another real number. \\
	The same thing could've been done in the opposite direction, showing that $\R$ has no maximum element, and that there exists a $z>x$ and a $y>z$ such that $x<z<y$ in the real numbers due to it's properties.
\end{proof}

\newtheorem*{prop8.49}{Proposition 8.49}
\begin{prop8.49}
	Let $A \subset \R$ be nonempty. If $\sup(A) \in A$ then $\sup(A)$ is the largest element of A, $\sup(A) = \max(A)$. Conversely, if A has a largest element then $\max(A) = \sup(A)$ and $\sup(A) \in A$.
\end{prop8.49}

\begin{proof}
	$b = \sup(A) \in A \implies b \in A$ and \\
	$\sup(A)$ is an upperbound for $A \implies b$ satisfies $a\in A \implies a \le b$ \\
	so $b = \max(A) = \sup(A)$.
	
	Conversely, if $b=\max(A)$ then $a\in A \implies a \le b$ by definition of a maximum. \\
	Also if $b'$ is an upperbound for A, $b \le b'$, because $b=\max(A) \implies b\in A$ and $b \le b'$ by definition of an upperbound. Therefore $\max(A) = \sup(A)$ and $sup(A) \in A$.
\end{proof}

\newtheorem*{prop8.50}{Proposition 8.50}
\begin{prop8.50}
	Suppose $A \subset B \subset \R$, A and B are bounded above. Then $\sup(A) \le \sup(B)$.
\end{prop8.50}

\begin{proof}
	$\sup(A)$ is a number $b$ such that $a \le b$ for all $a \in A$, and if $b'$ is an upper bound for A, then $b \le b'$.
	
	$A \subset B \implies$ every $a \in A$ is an element of B. \\
	$\sup(B)$ is an upper bound for $B \implies$ every element $c \in B$ follows $c \le \sup(B)$. \\
	So $\sup(B)$ is an upperbound for $A$ because every element of A is an element of B. \\
	Also, $\sup(A)$ is the least upper bound for A, so $\sup(A) \le \sup(B)$ by the definition of supremum.
\end{proof}

\newtheorem*{prop8.51}{Project 8.51}
\begin{prop8.51}
	For a nonempty set $B \subset \R$, one can define the greatest lower bound, $\inf(B)$ of B. Give the precise definition for $\inf(B)$ and prove that it is unique if it exists. Also define $\min(B)$ and prove the analogue of Proposition 8.49 for greatest lower bounds and minima.
\end{prop8.51}

\begin{proof}
	B is bounded below if there exists $a \in \R$ such that $b\in B \implies b \ge a$. \\
	The $\inf(B)$ for B is a lower bound $c$ such that if $c'$ is a lower bound, $c \ge c'$. \\
	If $\inf(B)$ exists, let $x_1$ and $x_2$ be greatest lower bounds for B. \\
	$x_1$ is a lower bound and $x_2$ is a greatest lower bound $\implies x_2 \ge x_1$. \\
	$x_2$ is a lower bound and $x_1$ is a greatest lower bound $\implies x_1 \ge x_2$. \\
	Therefore, $x_1 \le x_2 \le x_1$ \\
	So $x_1 = x_2$ by Proposition 8.31 so the greatest lower bound, or $\inf(B)$ is unique if it exists. \\
	
	$\min(B)$: An element $a \in B$ is the minimum/smallest element ($\min(B)$) of B if for all $b\in B$, $b \in B \implies a \le b$. \\
	
	Analogue to Proposition 8.49: \\
	$a = \inf(B) \in B \implies a \in B$. \\
	$\inf(B)$ is a lowerbound for B $\implies a$ satisfies $b \in B \implies a \le b$. \\
	So $a = \min(B)$. \\
	
	Conversely, assuming there exists $a = \min(B) \in R$. \\
	$b\in B \implies a \le b$ by definition of a minimum. \\
	$a = \min(B) \implies a \in B \implies a \ge a'$ for a lower bound $a'$, by definition of a lower bound. Therefore, $a = \inf(B) \in B$. So, $\min(B) = \inf(B) \in B$.
\end{proof}

\newtheorem*{prop8.53}{Proposition 8.53}
\begin{prop8.53}
	Every nonempty subset of $\R$ that is bounded below has a greatest lower bound.
\end{prop8.53}

\begin{proof} %s=x, t=y u=m, m=z 
	Let X be every nonempty subset of $\R$ that is bounded below and has a greatest lower bound. \\
	Let Y be the set {y = -x : $x \in X$}. Since X is bounded below, there exists a lower bound $z$ such that $z \le x$ for all $x \in X$. Therefore since $z \le x$, by subtracting x from both sides and subtracting z from both sides, we get $-x \le -z$. \\
	So $y \le -z$ for all $y \in Y$. So Y is bounded above by $-z$, so by Axiom 8.52, $\sup(Y)$ exists. \\
	Let $m = \sup(Y)$. Then we need to show that $-m = \inf(X)$. \\
	
	$-m \le x, \forall x \in X$ by Axiom 8.52 \\
	$y \le x, \forall x \in X \implies y \le -m$. \\
	
	Since $m$ is the least upper bound of Y, \\
	$-x \le m, \forall x \in X$ and \\
	$-x \le -y, \forall x \in X \implies m \le -y$. \\
	So $-x \le -y, \forall x \in X \implies m \le -y$. \\
	Therefore, $y \le x, \forall x \in X \implies y \le -m$. \\
	So $-m = \inf(X)$.
\end{proof}

\end{document}
